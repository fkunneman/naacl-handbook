This article attempts to place dependency annotation options on a solid theoretical and applied footing. By verifying the validity of some basic choices of the current dependency reference framework, Universal Dependencies (UD), in a perspective of general annotation principles, we show how some choices can lead to inconsistencies and discontinuities, partly due to UD's alternation between syntax and semantics. For some constructions, we propose better suited alternative structures with a clear-cut distinction of syntax and semantics. We propose a classification of conception-oriented, annotator-oriented, and finally, treebank end-user-oriented considerations to be used in the creation of new annotation schemes.
