Current automatic machine translation systems require heavy human proof-reading to produce high-quality translations. We present a new interactive machine translation approach aimed at providing a natural collaboration between humans and translation systems. As such, we grant the user complete freedom to validate and correct any part of the translations suggested by the system. Our approach is then designed according to the requirements placed by this unrestricted proof-reading protocol. In particular, the ability of the system to suggest new translations coherent with the set of potentially disjoint translation segments validated by the user. We evaluate our approach in a user-simulated setting where reference translations are considered the output desired by a human expert. Results show important reductions in the number of edits in comparison to decoupled post-editing and conventional prefix-based interactive translation prediction. Additionally, we provide evidence that it can also reduce the cognitive overload reported for interactive translation systems in previous user studies.
