Differentiating between outdated expressions and current expressions is not a trivial task for foreign language learners, and could be beneficial for lexicographers, as they examine expressions. Assuming that the usage of expressions over time can be represented by a time-series of their periodic frequencies over a large lexicographic corpus, we test the hypothesis that there exists an old-new relationship between the time-series of some synonymous expressions, a hint that a later expression has replaced an earlier one. Another hypothesis we test is that MWEs can be simply characterized by sparsity \& frequency thresholds. In order to find trends in MWE usage, we search for Multiword Expressions (MWEs) of 2-3 words in English that match a ready-made list of MWEs, within a dataset that is based on 1 million books. Using a historical thesaurus, we manually identify synonyms of expressions with the most positive, or negative, trends, use the resulting expression pairs to identify and visualize the relationship between synonymous expressions. As hypothesized, we found evidence that old-new relationships do exist between expressions, and that candidate expressions can be found by simple means, when we take the history of an expression into account.
