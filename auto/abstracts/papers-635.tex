Computationally modeling the evolution of science by tracking how scientific topics rise and fall over time has important implications for research funding and public policy. However, little is known about the mechanisms underlying topic growth and decline. We investigate the role of rhetorical framing: whether the rhetorical role or function that authors ascribe to topics (as methods, as goals, as results, etc.) relates to the historical trajectory of the topics. We train topic models and a rhetorical function classifier to map topic models onto their rhetorical roles in 2.4 million abstracts from the Web of Science from 1991-2010. We find that a topic's rhetorical function is highly predictive of its eventual growth or decline. For example, topics that are rhetorically described as results tend to be in decline, while topics that function as methods tend to be in early phases of growth.
