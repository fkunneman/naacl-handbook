The linguistic experiences of a person are an important part of their individuality. In this paper, we show that people can be modelled as vectors in a semantic space, using their personal interaction with specific language data. We also demonstrate that these vectors can be taken as representative of 'the kind of person' they are.  We build over 4000 speaker-dependent subcorpora using logs of Wikipedia edits, which are then used to build distributional vectors that represent individual speakers. We show that such 'person vectors' are informative to others, and they influence basic patterns of communication like the choice of one's interlocutor in conversation. Tested on an information-seeking scenario, where natural language questions must be answered by addressing the most relevant individuals in a community, our system outperforms a standard information retrieval algorithm by a considerable margin.
