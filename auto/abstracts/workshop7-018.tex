Despite considerable theoretical work in social sciences, ready to use resources are very limited compared to digitally available mass media resources. Thus, this project creates a political protest database from online news resources in Brazil that will be used to explain Brazilian welfare state policy changes. In this paper we present the preliminary results of a system that automatically crawls digital resources and produces a protest database, which includes events such as strikes, rallies, boycotts, protests, and riots, as well as their attributes such as location, participants, and ideology.
