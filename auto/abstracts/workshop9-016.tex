Studies have shown that Twitter can be used for health surveillance, and personal experience tweets (PETs) are an important source of information for health surveillance. To mine Twitter data requires a relatively balanced corpus and it is challenging to construct such a corpus due to the labor-intensive annotation tasks of large data sets. We developed a bootstrap method of finding PETs with the use of the machine learning-based filter. Through a few iterations, our approach can efficiently improve the balance of two class dataset with a reduced amount of annotation work. To demonstrate the usefulness of our method, a PET corpus related to effects caused by 4 dietary supplements was constructed. In 3 iterations, a corpus of 8,770 tweets was obtained from 108,528 tweets collected, and the imbalance of two classes was significantly reduced from 1:31 to 1:3. In addition, two out of three classifiers used showed improved performance over iterations. It is conceivable that our approach can be applied to various other health surveillance studies that use machine-learning based classifications of imbalanced Twitter data.
