We introduce a new language learning setting relevant to building adaptive natural language interfaces. It is inspired by Wittgenstein's language games: a human wishes to accomplish some task (e.g., achieving a certain configuration of blocks), but can only communicate with a computer, who performs the actual actions (e.g., removing all red blocks). The computer initially knows nothing about language and therefore must learn it from scratch through interaction, while the human adapts to the computer's capabilities. We created a game in a blocks world and collected interactions from 100 people playing it.  First, we analyze the humans' strategies, showing that using compositionality and avoiding synonyms correlates positively with task performance. Second, we compare computer strategies, showing how to quickly learn a semantic parsing model from scratch, and that modeling pragmatics further accelerates learning for successful players.
