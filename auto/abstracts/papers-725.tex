In conversation, speakers tend to ``accommodate'' or ``align'' to their partners, changing the style and substance of their communications to be more similar to their partners' utterances. We focus here on ``linguistic alignment,'' changes in word choice based on others' choices. Although linguistic alignment is observed across many different contexts and its degree correlates with important social factors such as power and likability, its sources are still uncertain. We build on a recent probabilistic model of alignment, using it to separate out alignment attributable to words versus word categories. We model alignment in two contexts: telephone conversations and microblog replies. Our results show evidence of alignment, but it is primarily lexical rather than categorical. Furthermore, we find that discourse acts modulate alignment substantially. This evidence supports the view that alignment is shaped by strategic communicative processes related to the ongoing discourse.
