This study aims at determining whether collocational features automatically extracted from EFL (English as a foreign language) texts are useful for quality scoring, and allow the improvement of a competitive baseline based on, amongst other factors, bigram frequencies. The collocational features were gathered by assigning to each bigram in an EFL text eight association scores computed on the basis of a native reference corpus. The distribution of the association scores were then summarized by a few global statistical features and by a discretizing procedure. An experiment conducted on a publicly available dataset confirmed the effectiveness of these features and the benefit brought by using several discretized association scores.
