One of the characteristics of child-directed speech is its high degree of repetitiousness. Sequences of repetitious utterances with a constant intention, variation sets, have been shown to be correlated with children's language acquisition. To obtain a baseline for the occurrences of variation sets in Swedish, we annotate 18 parent--child dyads using a generalised definition according to which the varying form may pertain not just to the wording but also to prosody and/or non-verbal cues. To facilitate further empirical investigation, we introduce a surface algorithm for automatic extraction of variation sets which is easily replicable and language-independent. We evaluate the algorithm on the Swedish gold standard, and use it for extracting variation sets in Croatian, English and Russian. We show that the proportion of variation sets in child-directed speech decreases consistently as a function of children's age across Swedish, Croatian, English and Russian.
