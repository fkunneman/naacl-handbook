This paper describes the Universitat d'Alacant submissions (labeled as UAlacant) to the machine translation quality estimation (MTQE) shared task at WMT 2016, where we have participated in the word-level and phrase-level MTQE sub-tasks. Our systems use external sources of bilingual information as a ``black box'' to spot sub-segment correspondences between the source segment and the translation hypothesis. For our submissions, two sources of bilingual information have been used: machine translation (Lucy LT KWIK Translator and Google Translate) and the bilingual concordancer Reverso Context. Building upon the word-level approach implemented for WMT 2015, a method for phrase-based MTQE is proposed which builds on the probabilities obtained for word-level MTQE. For each sub-task we have submitted two systems: one using the features produced exclusively based on on-line sources of bilingual information, and one combining them with the baseline features provided by the organisers of the task.
