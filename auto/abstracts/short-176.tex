Dimensional sentiment analysis aims to recognize continuous numerical values in multiple dimensions such as the valence-arousal (VA) space. Compared to the categorical approach that focuses on sentiment classification such as binary classification (i.e., positive and negative), the dimensional approach can provide more fine-grained sentiment analysis. This study proposes a regional CNN-LSTM model consisting of two parts: regional CNN and LSTM to predict the VA ratings of texts. Unlike a conventional CNN which considers a whole text as input, the proposed regional CNN uses an individual sentence as a region, dividing an input text into several regions such that the useful affective information in each region can be extracted and weighted according to their contribution to the VA prediction. Such regional information is sequentially integrated across regions using LSTM for VA prediction. By combining the regional CNN and LSTM, both local (regional) information within sentences and long-distance dependency across sentences can be considered in the prediction process. Experimental results show that the proposed method outperforms lexicon-based, regression-based, and NN-based methods proposed in previous studies.
