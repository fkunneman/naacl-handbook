Wikipedia is a resource of choice exploited in many NLP applications, yet we are not aware of recent attempts to adapt coreference resolution to this resource. In this work, we revisit a seldom studied task which consists in identifying in a Wikipedia article all the mentions of the main concept being described.  We show that by exploiting the Wikipedia markup of a document, as well as links to external knowledge bases such as Freebase, we can  acquire useful information on entities that helps to classify mentions as coreferent or not. We designed a classifier which drastically outperforms fair baselines built on top of state-of-the-art coreference resolution systems. We also measure the benefits of this classifier in a full coreference resolution pipeline applied to Wikipedia texts.
