Researchers of language variation and change often need to go to great lengths to find sufficient data, particularly when they shall be used for a sound statistical analysis of the phenomenon in question. The recent analogical change in the formation of the imperative singular of German strong verbs with vowel gradation is a case in point, as it could not have been studied without the compilation of a web-based corpus. On the one hand, the investigation was faced with a number of challenges during the compilation of the corpus, the search for relevant hits and their annotation for a number of variables. On the other hand, results which would otherwise not have been obtained balance out this increased amount of manual labour. The present paper elaborates on some of these challenges and provides suggestions how they might be avoided in similar investigations in future. It concludes by presenting invaluable insights which would not have been gained without the present corpus study.
