Content can be expressed at different lev- els of specificity, varying the amount of detail presented to the reader. The need to transform specific content into more general form naturally arises in summarization, where people and machines need to convey the gist of a text within im- posed space constraints. Completely re- moving sentences and phrases is one way to reduce the level of detail. The bulk of work on summarization content selection and compression deal with these tasks. In this paper, we present a cor- pus study on a more subtle and under- studied phenomenon: noun phrase generalization. Based on multi-document news and abstract alignments at the phrase level, we arrive at a five category classification scheme and find that the most common category requires semantic interpretation and inference. The others rely on lexical substitution or deletion of details from the original expression. We provide a system- atic analysis, elucidating the capabilities needed for automating the generation of more general or more specific references.
