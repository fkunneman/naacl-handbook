The applicability of entropy rate constancy to dialogue is examined on two spoken dialogue corpora. The principle is found to hold; however, new entropy change patterns within the topic episodes of dialogue are described, which are different from written text. Speaker's dynamic roles as topic initiators and topic responders are associated with decreasing and increasing entropy, respectively, which results in lo- cal convergence between these speakers in each topic episode. This implies that the sentence entropy in dialogue is conditioned on different contexts determined by the speaker's roles. Explanations from the perspectives of grounding theory and interactive alignment are discussed, resulting in a novel, unified information- theoretic approach of dialogue.
