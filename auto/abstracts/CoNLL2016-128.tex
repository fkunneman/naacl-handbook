Discourse relations can either be implicit or explicitly expressed by markers, such as 'therefore' and 'but'. How a speaker makes this choice is a question that is not well understood. We propose a psycholinguistic model that predicts whether a speaker will produce an explicit marker given the discourse relation s/he wishes to express. Based on the framework of the Rational Speech Acts model, we quantify the utility of producing a marker based on the information-theoretic measure of surprisal, the cost of production, and a bias to maintain uniform information density throughout the utterance. Experiments based on the Penn Discourse Treebank show that our approach outperforms state-of- the-art approaches, while giving an explanatory account of the speaker's choice.
