This paper provides a description of the automatic conversion of the morphologically annotated part of the Old Hungarian Corpus. These texts are in the format of the Humor analyzer, which does not follow any international standards. Since standardization always facilitates future research, even for researchers who do not know the Old Hungarian language, we opted for mapping the Humor formalism to a widely used universal tagset, namely the Universal Dependencies framework. The benefits of using a shared tagset across languages enable interlingual comparisons from a theoretical point of view and also multilingual NLP applications can profit from a unified annotation scheme. In this paper, we report the adaptation of the Universal Dependencies morphological annotation scheme to Old Hungarian, and we discuss the most important theoretical linguistic issues that had to be resolved during the process. We focus on the linguistic phenomena typical of Old Hungarian that required special treatment and we offer solutions to them.
