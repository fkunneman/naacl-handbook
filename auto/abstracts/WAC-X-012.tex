In this paper, we describe preliminary results from an ongoing experiment wherein we classify two large unstructured text corpora --- a web corpus and a newspaper corpus --- by topic domain (or subject area). Our primary goal is to develop a method that allows for the reliable annotation of large crawled web corpora with meta data required by many corpus linguists. Since we use data from a web corpus and a more traditional corpus, we also contribute to the important field of corpus comparison and corpus evaluation. Technically, we use (unsupervised) topic modelling (LSI and LDA) to automatically induce topic distributions over gold standard corpora that were manually annotated for 13 coarse-grained topic domains. In a second step, we apply supervised machine learning (SVMs) to learn the manually annotated topic domains using the previously induced topics as features. We achieve around 70\% accuracy in 10-fold cross validations. An analysis of the errors clearly indicates, however, that a revised classification scheme and larger gold standard corpora will likely lead to a dramatic increase in accuracy.
