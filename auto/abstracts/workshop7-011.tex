Sarcasm annotation extends beyond linguistic expertise, and often involves cultural context. This paper presents our first-of-its-kind study that deals with impact of cultural differences on the quality of sarcasm annotation. For this study, we consider the case of American text and Indian annotators. For two sarcasm-labeled datasets of American tweets and discussion forum posts that have been annotated by American annotators, we obtain annotations from Indian annotators. Our Indian annotators agree with each other more than their American counterparts, and face difficulties in case of unfamiliar situations and named entities.  However, these difficulties in sarcasm annotation result in statistically insignificant degradation in sarcasm classification. We also show that these disagreements between annotators can be predicted using textual properties. Although the current study is limited to two annotators and one culture pair, our paper opens up a novel direction in evaluation of the quality of sarcasm annotation, and the impact of this quality on sarcasm classification.
