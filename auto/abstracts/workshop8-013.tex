Differential diagnosis aims at distinguishing between diseases causing similar symptoms. This is exemplified by epilepsies and dissociative disorders. Recently, it has been shown that linguistic features of physician-patient talks allow for differentiating between these two diseases. Since this method relies on trained linguists, it is not suitable for daily use. In this paper, we introduce a novel approach, called text2voronoi, for utilizing the paradigm of text visualization to reconstruct differential diagnosis as a task of text categorization. In line with current research on linguistic differential diagnosis, we explore linguistic characteristics of physician-patient talks to span our feature space. However, unlike standard approaches to categorization, we do not use linguistic feature spaces directly, but explore visual features derived from the talks' pictorial representations. That is, we provide an approach to image-driven differential diagnosis. By example of 24 talks of epileptics and dissociatively disordered patients, we show that our approach outperforms its counterpart based on the bag-of-words model.
