We focus on two leading state-of-the-art approaches to grammatical error correction --- machine learning classification and machine translation. Based on the comparative study of the two learning frameworks and through error analysis of the output of the state-of-the-art systems, we identify key strengths and weaknesses of each of these approaches and demonstrate their complementarity. In particular, the machine translation method learns from parallel data without requiring further linguistic input and is better at correcting complex mistakes. The classification approach possesses other desirable characteristics, such as the ability to easily generalize beyond what was seen in training, the ability to train without human-annotated data, and the flexibility to adjust knowledge sources for individual error types. Based on this analysis, we develop an algorithmic approach that combines the strengths of both methods. We present several systems based on resources used in previous work with a relative improvement of over 20\% (and 7.4 F score points) over the previous state-of-the-art.
