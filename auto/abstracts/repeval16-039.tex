The way humans define words is a powerful way of representing them.  In this work, we propose to measure word similarity by comparing the overlap in their definition.  This highlights linguistic phenomena that are complementary to the information extracted from standard context-based representation learning techniques.  To acquire a large amount of word definitions in a cost-efficient manner, we designed a simple interactive word game, Word Sheriff.  As a byproduct of game play, it generates short word sequences that can be used to uniquely identify words.  These sequences can not only be used to evaluate the quality of word representations, but it could ultimately give an alternative way of learning them, as it overcomes some of the limitations of the distributional hypothesis.  Moreover, inspecting player behaviour reveals interesting aspects about human strategies and knowledge acquisition beyond those of simple word association games, due to the conversational nature of the game.  Lastly, we outline a vision of a communicative evaluation setting, where systems are evaluated based on how well a given representation allows a system to communicate with human and computer players.
