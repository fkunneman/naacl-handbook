We explore two approaches to model complement types (NPs and PPs) in an English-to-German SMT system: A simple abstract representation inserts pseudo-prepositions that mark the beginning of noun phrases, to improve the symmetry of source and target complement types, and to provide a flat structural information on phrase boundaries. An extension of this representation generates context-aware synthetic phrase-table entries conditioned on the source side, to model complement types in terms of grammatical case and preposition choice. Both the simple preposition-informed system and the context-aware system significantly improve over the baseline; and the context-aware system is slightly better than the system without context information.
