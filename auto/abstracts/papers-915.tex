The field of grammatical error correction (GEC) has grown substantially in recent years, with research directed at both evaluation metrics and improved system performance against those metrics. One unvisited assumption, however, is the reliance of GEC evaluation on error-coded corpora, which contain specific labeled corrections. We examine current practices and show that GEC's reliance on such corpora unnaturally constrains annotation and automatic evaluation, resulting in (a) sentences that do not sound acceptable to native speakers and (b) system rankings that do not correlate with human judgments. In light of this, we propose an alternate approach that jettisons costly error coding in favor of unannotated, whole-sentence rewrites. We compare the performance of existing metrics over different gold-standard annotations, and show that automatic evaluation with our new annotation scheme has very strong correlation with expert rankings (ρ = 0.82). As a result, we advocate for a fundamental and necessary shift in the goal of GEC, from correcting small, labeled error types, to producing text that has native fluency.
