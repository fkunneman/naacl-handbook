Hand-crafted features based on linguistic and domain-knowledge play crucial role in determining the performance of disease name recognition systems. Such methods are further limited by the scope of these features or in other words, their ability to cover the contexts or word dependencies within a sentence. In this work, we focus on reducing such dependencies and propose a domain-invariant framework for the disease name recognition task. In particular, we propose various end-to-end recurrent neural network (RNN) models with conditional random fields (CRF) for the tasks of disease name recognition and their classification into four pre-defined categories. We also utilize convolution neural network (CNN) in cascade of RNN to get character-based embedded features and employ it with word-embedded features in our model. We compare our models with the state-of-the-art results for the two tasks on NCBI disease dataset. Our results for the disease mention recognition task indicate that state-of-the-art performance can be obtained without relying on feature engineering. Further our results significantly improved the state-of-the-art performance for the classification task of disease names.
