We consider two graph models of semantic change. The first is a time-series model that relates embedding vectors from one time period to embedding vectors of previous time periods. In the second, we construct one graph for each word: nodes in this graph correspond to time points and edge weights to the similarity of the word's meaning across two time points. We apply our two models to corpora across three different languages. We find that semantic change is linear in two senses. Firstly, today's embedding vectors (= meaning) of words can be derived as linear combinations of embedding vectors of their neighbors in previous time periods. Secondly, self-similarity of words decays linearly in time. We consider both findings as new laws/hypotheses of semantic change.
