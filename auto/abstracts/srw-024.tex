This paper evaluates the challenges in- volved in shallow parsing of Dravidian languages which are highly agglutinative and morphologically rich. Text processing tasks in these languages are not trivial be- cause multiple words concatenate to form a single string with morpho-phonemic changes at the point of concatenation. This phenomenon known as Sandhi, in turn complicates the individual word identifi- cation. Shallow parsing is the task of identification of correlated group of words given a raw sentence. The current work is an attempt to study the effect of Sandhi in building shallow parsers for Dravidian lan- guages by evaluating its effect on Malay- alam, one of the main languages from Dra- vidian family. We provide an in-depth analysis of effect of Sandhi in developing a robust shallow parser pipeline with exper- imental results emphasizing on how sen- sitive the individual components of shal- low parser are, towards the accuracy of a sandhi splitter. Our work can serve as a guiding light for building robust text pro- cessing systems in Dravidian language.
