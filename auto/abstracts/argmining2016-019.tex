The annotation of argument schemes represents an important step for argumentation mining. General guidelines for the annotation of argument schemes, applicable to any topic, are still missing due to the lack of a suitable taxonomy in Argumentation Theory and the need for highly trained expert annotators. We present a set of guidelines for the annotation of argument schemes, taking as a framework the Argumentum Model of Topics (Rigotti and Morasso, 2010; Rigotti, 2009). We show that this approach can contribute to solving the theoretical problems, since it offers a hierarchical and finite taxonomy of argument schemes as well as systematic, linguistically-informed criteria to distinguish various types of argument schemes. We describe a pilot annotation study of 30 persuasive essays using multiple minimally trained non-expert annotators .Our findings from the confusion matrixes pinpoint problematic parts of the guidelines and the underlying annotation of claims and premises. We conduct a second annotation with refined guidelines and trained annotators on the 10 essays which received the lowest agreement initially. A significant improvement of the inter-annotator agreement shows that the annotation of argument schemes requires highly trained annotators and an accurate annotation of argumentative components (premises and claims).
