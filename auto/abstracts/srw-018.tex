This paper presents an application of the meta-bootstrapping algorithm from Riloff and Jones 1999 as a means to handle difficulties associated with non-standard language usage in the social media domain when tweets were analyzed for potentially containing adverse drug events. The paper details the importance of adverse event detection, presents a large new dataset, and steps through several different evaluations exploring the effects of incrementing the amount of unannotated social media data given to the bootstrapping procedure. Modifications of the original algorithm are discussed in addition to highlighting several important examples of how meta-bootstrapped lexicons can handle internet language use within the clinical domain in a way studies using medical ontologies cannot.
