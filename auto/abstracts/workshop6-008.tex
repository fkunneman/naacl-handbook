Recent work has proposed using network science to analyze the structure of the mental lexicon by viewing words as nodes in a phonological network, with edges connecting words that differ by a single phoneme. Comparing the structure of phonological networks across different languages could provide insights into linguistic typology and the cognitive pressures that shape language acquisition, evolution, and processing. However, previous studies have not considered how statistics gathered from these networks are affected by factors such as lexicon size and the distribution of word lengths. We show that these factors can substantially affect the statistics of a phonological network and propose a new method for making more robust comparisons. We then analyze eight languages, finding many commonalities but also some qualitative differences in their lexicon structure.
