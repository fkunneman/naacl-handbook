Experiments in Artificial Language Learning have revealed much about the cognitive mechanisms underlying sequence and language learning in human adults, in infants and in non-human animals. This paper focuses on their ability to generalize to novel grammatical instances (i.e., instances consistent with a familiarization pattern). Notably, the propensity to generalize appears to be negatively correlated with the amount of exposure to the artificial language, a fact that has been claimed to be contrary to the predictions of statistical models (Peña et al. (2002); Endress and Bonatti (2007)). In this paper, we propose to model generalization as a threestep process, and we demonstrate that the use of statistical models for the first two steps, contrary to widespread intuitions in the ALL-field, can explain the observed decrease of the propensity to generalize with exposure time.
