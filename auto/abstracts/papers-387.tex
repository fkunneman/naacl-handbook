To establish sophisticated dialogue systems, text planning needs to cope with congruent as well as incongruent interlocutor interests as given in everyday dialogues. Little attention has been given to this topic in text planning in contrast to dialogues that are fully aligned with anticipated user interests. When considering dialogues with congruent and incongruent interlocutor interests, dialogue partners are facing the constant challenge of finding a balance between cooperation and competition. We introduce the concept of fairness that operationalize an equal and adequate, i.e. equitable satisfaction of all interlocutors' interests. Focusing on Question-Answering (QA) settings, we describe an answer planning approach that support fair dialogues under congruent and incongruent interlocutor interests. Due to the fact that fairness is subjective per se, we present promising results from an empirical study (N=107) in which human subjects interacted with a QA system implementing the proposed approach.
