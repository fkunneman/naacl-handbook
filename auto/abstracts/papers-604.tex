Properties of corpora, such as the diversity of vocabulary and how tightly related texts cluster together, impact the best way to cluster short texts. We examine several such properties in a variety of corpora and track their effects on various combinations of similarity metrics and clustering algorithms. We show that semantic similarity metrics outperform traditional \$n\$-gram and dependency similarity metrics for k-means clustering of a linguistically creative dataset, but do not help with less creative texts. Yet the choice of similarity metric interacts with the choice of clustering method. We find that graph-based clustering methods perform well on tightly clustered data but poorly on loosely clustered data. Semantic similarity metrics generate loosely clustered output even when applied to a tightly clustered dataset. Thus, the best performing clustering systems could not use semantic metrics.
