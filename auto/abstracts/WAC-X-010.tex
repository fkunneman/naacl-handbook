Metadata extraction is known to be a problem in general-purpose Web corpora, and so is extensive crawling with little yield. The contributions of this paper are threefold:{\textasciitilde}a method to find and download large numbers of WordPress pages; a targeted extraction of content featuring much needed metadata; and an analysis of the documents in the corpus with insights of actual blog uses. The study focuses on a publishing software (WordPress), which allows for reliable extraction of structural elements such as metadata, posts, and comments. The download of about 9 million documents in the course of two experiments leads after processing to 2.7 billion tokens with usable metadata. This comparatively high yield is a step towards more efficiency with respect to machine power and ``Hi-Fi'' web corpora. The resulting corpus complies with formal requirements on metadata-enhanced corpora and on weblogs considered as a series of dated entries. However, existing typologies on Web texts have to be revised in the light of this hybrid genre.
