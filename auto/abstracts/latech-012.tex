Syntactic change tends to affect constructions, but treebanks annotate lower-level structure: PCFG rules or dependency arcs. This paper extends prior work in native language identification, using Tree Substitution Grammars to discover constructions which can be tested for historical variability. In a case study comparing Classical and Medieval Latin, the system discovers several constructions corresponding to known historical differences, and learns to distinguish the two varieties with high accuracy. Applied to an intermediate text (the Vulgate Bible), it indicates which changes between the eras were already occurring at this earlier stage.
