Arab users sometimes tend to express their natural dialectal Arabic voice in text using a Latin script on social media. This form of transliterated Arabic is called Arabizi. Extracting sentiment from Arabizi is considered complicated for the following reasons: 1. Arabizi is dialectal, differing among regions. 2. Arabizi users do not follow a unified orthography nor grammar. 3. To the best of our knowledge, there are no public lexical resources to process this data. Several researchers who worked on sentiment analysis for Arabic filtered Arabizi from their datasets. As this data might contain valuable information, we first investigate how frequent is Arabizi on Twitter to show the amount of Arabizi data being filtered out. Second we study which methods could be used to create Arabizi identification tools. We believe that identifying Arabizi from multilingual data brings us a step forward towards  analysing sentiment from Arabic social media.
