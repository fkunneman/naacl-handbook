Genre and domain are well known covariates of both manual and automatic annotation quality. Comparatively less is known about the effect of sentence types, such as imperatives, questions or fragments, and how they interact with text type effects. Using mixed effects models, we evaluate the relative influence of genre and sentence types on automatic and manual annotation quality for three related tasks in English data: POS tagging, dependency parsing and coreference resolution. For the latter task, we also develop a new metric for the evaluation of individual regions of coreference annotation. Our results show that while there are substantial differences between manual and automatic annotation in each task, sentence type is generally more important than genre in predicting errors within our data.
