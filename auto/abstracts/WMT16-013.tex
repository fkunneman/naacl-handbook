Information about the antecedents of pronouns is considered essential to solve certain translation divergencies of pronouns, such as those concerning the English ``it'' when translated into gendered languages, e.g. for French into ``il'', ``elle'', or several other options.  However, none of the machine translation (MT) systems using anaphora resolution has been able to outperform a phrase-based statistical MT baseline.  We address here one of the reasons for this failure: the imperfection of automatic anaphora resolution algorithms.  We learn from parallel data probabilistic correlations between target-side pronouns and the gender and number features of their (uncertain) antecedents, as hypothesized by the Stanford Coreference Resolution system on the source side.  We embody these correlations into a secondary translation model, which we invoke upon decoding with the Moses statistical MT system.  This solution outperforms a deterministic pronoun post-editing system, as well as a statistical MT baseline, on automatic and human evaluation metrics.
