One may express favor (or disfavor) towards a target by using  positive or negative language. Here for the first time we present a dataset of tweets annotated for whether the tweeter is in favor of or against pre-chosen targets, as well as for sentiment. These targets may or may not be referred to in the tweets, and they may or may not be the target of opinion in the tweets. We develop a simple stance detection system that outperforms all 19 teams that participated in a recent shared task competition on the same dataset. Additionally, access to both stance and sentiment annotations allows us to conduct several experiments to tease out their interactions. We show that while sentiment features  are useful for stance classification, they alone are not sufficient. We also show the impacts of various features on detecting stance and sentiment, respectively.
