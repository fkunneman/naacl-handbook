Similarity is a core notion that is used in psychology and two branches of linguistics: theoretical and computational. The similarity datasets that come from the two fields differ in design: psychological datasets are focused around a certain topic such as fruit names, while linguistic datasets contain words from various categories. The later makes humans assign low similarity scores to the words that have nothing in common and to the words that have contrast in meaning, making similarity scores ambiguous. In this work we discuss the similarity collection procedure for a multi-category dataset that avoids score ambiguity and suggest changes to the evaluation procedure to reflect the insights of psychological literature for word, phrase and sentence similarity. We suggest to ask humans to provide a list of commonalities and differences instead of numerical similarity scores and employ the structure of human judgements beyond pairwise similarity for model evaluation. We believe that the proposed approach will give rise to datasets that test meaning representation models more thoroughly with respect to the human treatment of similarity.
