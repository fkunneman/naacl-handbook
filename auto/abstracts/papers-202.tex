Today's extraction of temporal information for events heavily depends on annotated temporal links. These so called TLINKs capture the relation between pairs of event mentions and time expressions. One problem is that the number of possible TLINKs grows quadratic with the number of event mentions, therefore most annotation studies concentrate on links for mentions in the same or in adjacent sentences. However, as our annotation study shows, this restriction results for 58\% of the event mentions in a less precise information when the event took place. This paper proposes a new annotation scheme to anchor events in time. Not only is the annotation effort much lower as it scales linear with the number of events, it also gives a more precise anchoring when the events have happened as the complete document can be taken into account. Using this scheme, we annotated a subset of the TimeBank Corpus and compare our results to other annotation schemes. Additionally, we present some baseline experiments to automatically anchor events in time.
