Part of the unique cultural heritage of China is the Chinese couplet. Given a sentence (namely an antecedent clause), people reply with another sentence (namely a subsequent clause) equal in length. Moreover, a special phenomenon is that corresponding characters from the same position in the two clauses match each other by following certain constraints on semantic and/or syntactic relatedness. Automatic couplet generation by computer is viewed as a difficult problem and has not been fully explored. In this paper, we formulate the task as a natural language generation problem using neural network structures. Given the issued antecedent clause, the system generates the subsequent clause via sequential language modeling. To satisfy special characteristics of couplets, we incorporate the attention mechanism and polishing schema into the encoding-decoding process. The couplet is generated incrementally and iteratively. A comprehensive evaluation, using perplexity and BLEU measurements as well as human judgments, has demonstrated the effectiveness of our proposed approach.
