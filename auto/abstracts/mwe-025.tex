This paper describes results of a study related to the PARSEME Shared Task on automatic detection of verbal Multi-Word Expressions (MWEs) which focuses on their identification in running texts in many languages. The Shared Task's organizers have provided basic annotation guidelines where four basic types of verbal MWEs are defined including some specific subtypes. Czech is among the twenty languages selected for the task. We will contribute to the Shared Task dataset,  a multilingual open resource, by converting data from the Prague Dependency Treebank (PDT) to the Shared Task format. The question to answer is to which extent this can be done automatically. In this paper, we concentrate on one of the relevant MWE categories, namely on the quasi-universal category called ``Inherently Pronominal Verbs'' (IPronV) and describe its annotation in the Prague Dependency Treebank. After comparing it to the Shared Task guidelines, we can conclude that the PDT and the associated valency lexicon, PDT-Vallex, contain sufficient information for the conversion, even if some specific instances will have to be checked.  As a side effect, we have identified certain errors in PDT annotation which can now be corrected.
