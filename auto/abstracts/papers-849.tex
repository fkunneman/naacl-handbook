Fill-in-the-blank items are commonly featured in computer-assisted language learning (CALL) systems. An item displays a sentence with a  blank, and often proposes a number of choices for filling it.  These choices should include one correct answer and several plausible distractors.  We describe a system that, given an English corpus, automatically generates distractors to produce items for preposition usage.                          We report a comprehensive evaluation on this system, involving both experts and learners.  First, we analyze the difficulty levels of machine-generated carrier sentences and distractors, comparing several methods that exploit learner error and learner revision patterns.  We show that the quality of machine-generated items approaches that of human-crafted ones. Further, we investigate the extent to which mismatched L1 between the user and the learner corpora affects the quality of distractors.  Finally, we measure the system's impact on the user's language proficiency in both the short and the long term. Further, we investigate the extent to which mismatching L1 between the user and the learner corpora affects the quality of distractors. Finally, we report the system's impact on learners' language proficiency in both the short and the long term.
