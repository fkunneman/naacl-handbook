Works on Twitter community detection have yielded new ways to extract valuable insights from social media. Through this technique, Twitter users can be grouped into different types of communities such as those who have the same interests, those who interact a lot, or those who have similar sentiments about certain topics. Computationally, information is represented as a graph, and community detection is the problem of partitioning the graph such that each community is more densely connected to each other than to the rest of the network. It has been shown that incorporating sentiment analysis can improve community detection when looking for sentiment-based communities. However, such works only perform sentiment analysis in isolation without considering the tweet's various contextual information. Examples of these contextual information are social network structure, and conversational, author, and topic contexts. Disregarding these information poses a problem because at times, context is needed to clearly infer the sentiment of a tweet. Thus, this research aims to improve detection of sentiment-based communities on Twitter by performing contextual sentiment analysis.
