Many studies have been made on the language alterations that take place over the course of Alzheimer's disease (AD). As a consequence, it is now admitted that it is possible to discriminate between healthy and ailing patients solely based on the analysis of language production. Most of these studies, however, were made on very small samples—30 participants per study, on an average—, or involved a great deal of manual work in their analysis. In this paper, we present an automatic analysis of transcripts of elderly participants describing six common objects. We used part-of-speech and lexical richness as linguistic features to train an SVM classifier to automatically discriminate between healthy and AD patients in the early and moderate stages. The participants, in the corpus used for this study, were 63 Spanish adults over 55 years old (29 controls and 34 AD patients). With an accuracy of 88\%, our experimental results compare favorably to those relying on the manual extraction of attributes, providing evidence that the need for manual analysis can be overcome without sacrificing in performance.
