Emojis are a quickly spreading and rather unknown communication phenomenon which occasionally receives attention in the mainstream press, but lacks the scientific exploration it deserves. This paper is a first attempt at investigating the global distribution of emojis. We perform our analysis of the spatial distribution of emojis on a dataset of {\textasciitilde}17 million (and growing) geo-encoded tweets containing emojis by running a cluster analysis over countries represented as emoji distributions and performing correlation analysis of emoji distributions and World Development Indicators. We show that emoji usage tends to draw quite a realistic picture of the living conditions in various parts of our world.
