Conversion is a word formation operation that changes the grammatical category of a word in the absence of overt morphology. Conversion is extremely productive in English (e.g., tunnel, talk). This paper investigates whether distributional information can be used to predict the diachronic direction of conversion for homophonous noun---verb pairs. We aim to predict, for example, that tunnel was used as a noun prior to its use as a verb. We test two hypotheses: (1) that derived forms are less frequent than their bases, and (2) that derived forms are more semantically specific than their bases, as approximated by information theoretic measures. We find that hypothesis (1) holds for N-to-V conversion, while hypothesis (2) holds for V-to-N conversion. We achieve the best overall account of the historical data by taking both frequency and semantic specificity into account. These results provide a new perspective on linguistic theories regarding the semantic specificity of derivational morphemes, and on the morphosyntactic status of conversion.
