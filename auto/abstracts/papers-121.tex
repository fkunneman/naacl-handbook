Arbitrariness of the sign---the notion that the forms of words are unrelated to their meanings---is an underlying assumption of many linguistic theories.  Two lines of research have recently challenged this assumption, but they produce differing characterizations of non-arbitrariness in language.  Behavioral and corpus studies have confirmed the validity of localized form-meaning patterns manifested in limited subsets of the lexicon.  Meanwhile, global (lexicon-wide) statistical analyses instead find diffuse form-meaning systematicity across the lexicon as a whole. We bridge the gap with an approach that can detect both local and global form-meaning systematicity in language.  In the kernel regression formulation we introduce, form-meaning relationships can be used to predict words' distributional semantic vectors from their forms.  Furthermore, we introduce a novel metric learning algorithm that can learn weighted edit distances that minimize kernel regression error.  Our results suggest that the English lexicon exhibits far more global form-meaning systematicity than previously discovered, and that much of this systematicity is focused in localized form-meaning patterns.
