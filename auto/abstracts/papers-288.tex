A fundamental problem in artificial intelligence is automatically generating a natural language description of an image. This task involves both computer vision and natural language processing and is called ``image caption generation.'' Research on image caption generation has typically focused on taking in an image and generating a caption in English as existing image caption corpora are mostly in English. The lack of corpora in languages other than English is an issue, especially for morphologically rich languages such as Japanese. There is thus a need for corpora sufficiently large for image captioning in other languages. We have developed a Japanese version of the MS COCO caption dataset and a generative model based on a deep recurrent architecture that takes in an image and uses this Japanese version of the dataset to generate a caption in Japanese. As the Japanese portion of the corpus is small, our model was designed to transfer the knowledge representation obtained from the English portion into the Japanese portion. Experiments showed that the resulting bilingual comparable corpus has better performance than a monolingual corpus, indicating that image understanding using a resource-rich language benefits a resource-poor language.
