This paper is a novel study that views sarcasm detection in dialogue as a sequence labeling task, where a dialogue is made up of a sequence of utterances. We create a manually-labeled dataset of dialogue from TV series `Friends' annotated with sarcasm. Our goal is to predict sarcasm in each utterance, using sequential nature of a scene. We show performance gain using sequence labeling as compared to classification-based approaches. Our experiments are based on three sets of features, one is derived from information in our dataset, the other two are from past works. Two sequence labeling algorithms (SVM-HMM and SEARN) outperform three classification algorithms (SVM, Naive Bayes) for all these feature sets, with an increase in F-score of around 4\%. Our observations highlight the viability of sequence labeling techniques for sarcasm detection of dialogue.
