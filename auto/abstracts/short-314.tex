In all previous work on deep multi-task learning we are aware of, all task supervisions are on the same (outermost) layer. We present a multi-task learning architecture with deep bi-directional RNNs, where different tasks supervision can happen at different layers. We present experiments in syntactic chunking and CCG supertagging, coupled with the additional task of POS-tagging. We show that it is consistently better to have POS supervision at the innermost rather than the outermost layer. We argue that this is because ``low-level'' tasks are better kept at the lower layers, enabling enabling the higher-level tasks make use of the shared representation of the lower-level tasks. Finally, we also show how this architecture can be used for domain adaptation.
