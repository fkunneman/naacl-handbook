Recent years have seen a dramatic growth in the popularity of word embeddings mainly owing to their ability to capture semantic information from massive amounts of textual content. As a result, many tasks in Natural Language Processing have tried to take advantage of the potential of these distributional models. In this work, we study how word embeddings can be used in Word Sense Disambiguation, one of the oldest tasks in Natural Language Processing and Artificial Intelligence. We propose different methods through which word embeddings can be leveraged in a state-of-the-art supervised WSD system architecture, and perform a deep analysis of how different parameters affect performance. We show how a WSD system that makes use of word embeddings alone, if designed properly, can provide significant performance improvement over a state-of-the-art WSD system that incorporates several standard WSD features.
