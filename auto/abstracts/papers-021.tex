Cognitive science researchers have emphasized the importance of ordering a complex task into a sequence of easy to hard problems. Such an ordering provides an easier path to learning and increases the speed of acquisition of the task compared to conventional learning. Recent works in machine learning have explored a curriculum learning approach called self-paced learning which orders data samples on the easiness scale so that easy samples can be introduced to the learning algorithm first and harder samples can be introduced successively. We explore curriculum learning in the context of non-convex models for question answering and show that these approaches lead to improvements. We argue that incorporating easy, yet, a diverse set of questions, which are different from the questions already seen by the learner, can further improve learning. Our experiments support this hypothesis.
