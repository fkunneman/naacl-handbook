In recent years, social media has revolutionized how people communicate and share in-formation. One function of social media, besides connecting with friends, is sharing opinions with others. Microblogging sites, like Twitter, have often provided an online forum for social activism. When users debate about controversial topics on social media, they typically share different types of evidence to support their claims. Classifying these types of evidence can provide an estimate for how adequately the arguments have been supported. We first introduce a manually built gold standard dataset of 3000 tweets related to the recent FBI and Apple encryption debate. We develop a framework for automatically classifying six evidence types typically used on Twitter to discuss the debate. Our findings show that a Support Vector Machine (SVM) classifier trained with n-gram and additional features is capable of capturing the different forms of representing evidence on Twitter, and exhibits significant improvements over the unigram baseline, achieving an F1 macro-averaged of 82.8\%.
