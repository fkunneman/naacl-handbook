Recently, researchers in speech recognition have started to reconsider using whole words as the basic modeling unit, instead of phonetic units. These systems rely on a function that embeds an arbitrary or fixed dimensional speech segments to a vector in a fixed-dimensional space, named acoustic word embedding. Thus, speech segments of words that sound similarly will be projected in a close area in a continuous space. This paper focuses on the evaluation of acoustic word embeddings. We propose two approaches to evaluate the intrinsic performances of acoustic word embeddings in comparison to orthographic representations in order to evaluate whether they capture discriminative phonetic information. Since French language is targeted in experiments, a particular focus is made on homophone words.
