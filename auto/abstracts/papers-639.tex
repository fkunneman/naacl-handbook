The automatic detection of causal relationships in text is important for natural language understanding. This task has proven to be difficult, however, due to the need for world knowledge and inference. We focus on a sub-task of this problem where an open class set of linguistic markers can provide clues towards understanding causality. Unlike the explicit markers, a closed class, these markers vary significantly in their linguistic forms. We leverage parallel Wikipedia corpora to identify new markers that are variations on known causal phrases, creating a training set via distant supervision. We also train a causal classifier using features from the open class markers and semantic features providing contextual information. The results show that our features provide an 11.05 point absolute increase over the baseline on the task of identifying causality in text.
