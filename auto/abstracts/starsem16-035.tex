The interpretation of adjective-noun pairs plays a crucial role in tasks such as recognizing textual entailment. Formal semantics often places adjectives into a taxonomy which should dictate adjectives' entailment behavior when placed in adjective-noun compounds.  However, we show experimentally that the behavior of subsective adjectives (e.g. red) versus non-subsective adjectives (e.g. fake) is not as cut and dry as often assumed. For example, inferences are not always symmetric: while ID is generally considered to be mutually exclusive with fake ID, fake ID is considered to entail ID. We discuss the implications of these findings for automated natural language understanding.
