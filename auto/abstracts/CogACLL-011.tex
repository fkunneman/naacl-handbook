Many studies have found that language alterations can aid in the detection of certain medical afflictions. In this work, we present an ongoing project for recollecting multilingual conversations with the elderly in Latin America. This project, so far, involves the combined efforts of psychogeriatricians, linguists, computer scientists, research nurses and geriatric caregivers from six institutions across USA, Canada, Mexico and Ecuador. The recollections are being made available to the international research community. They consist of conversations with adults aged sixty and over, with different nationalities and socio-economic backgrounds. Conversations are recorded on video, transcribed and time-aligned. Additionally, we are in the process of receiving written texts—recent or old—authored by the participants, provided voluntarily. Each participant is recorded at least twice a year to allow longitudinal studies. Furthermore, information such as medical history, educational background, economic level, occupation, medications and treatments is being registered to aid conducting research on treatment progress and pharmacological effects. Potential studies derived from this work include speech, voice, writing, discourse, and facial and corporal expression analysis. We believe that our recollections incorporate complementary data that can aid researchers in further understanding the progression of cognitive degenerative diseases of the elderly.
