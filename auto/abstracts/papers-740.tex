We examine communications in a social network to study user emotional contrast - the propensity of users to express different emotions than those expressed by their neighbors. Our analysis is based on a large Twitter dataset, consisting of the tweets of 123,513 users from the USA and Canada. Focusing on Ekman's basic emotions, we analyze differences between the emotional tone expressed by these users and their neighbors and correlate these differences with perceived user demographics. We demonstrate that many perceived user demographic traits correlate with the emotional contrast between users and their neighbors. Unlike other approaches on inferring user demographics in social media that rely solely on user communications, we explore the network structure and show that it is possible to accurately predict a range of perceived demographic traits based solely on the emotions emanating from users and their neighbors.
