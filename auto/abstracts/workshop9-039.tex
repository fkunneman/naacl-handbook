Time-sensitive communication of critical imaging findings like pneumothorax or pulmonary embolism to referring physicians is important for patient safety. However, radiology findings are recorded in free-text format, relying on verbal communication that is not always successful. Natural language processing can provide automated suggestions to radiologists that new critical findings be added to a followup list. We present a pilot assessment of the feasibility of an automated critical finding suggestion system for radiology reporting by assessing suggestions made by the pyConTextNLP algorithm. Our evaluation focused on the false alarm rate to determine feasibility of deployment without increasing alert fatigue. pyConTextNLP identified 77 critical findings from 1,370 chest exams. Review of the suggested findings demonstrated a 7.8\% false alarm rate. We discuss the errors, which would be challenging to address, and compare pyConTextNLP's false alarm rate to false alarm rates of similar systems from the literature.
