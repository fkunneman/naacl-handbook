Publicly available knowledge repositories, such as Wikipedia and Freebase, benefit significantly from volunteers, whose contributions ensure that the knowledge keeps expanding and is kept up-to-date and accurate. User interactions are often limited to hypertext, tabular, or graph visualization interfaces. For spatio-temporal information, however, other interaction paradigms may be better-suited. We present an integrated system that combines crowdsourcing, automatic or semi-automatic knowledge harvesting from text, and visual analytics. It enables users to analyze large quantities of structured data and unstructured textual data from a spatio-temporal perspective and gain deep insights that are not easily observed in individual facts.
