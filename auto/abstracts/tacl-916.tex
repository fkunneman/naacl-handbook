Continuous word representations have been remarkably useful across NLP tasks but remain poorly understood. We ground word embeddings in semantic spaces studied in the cognitive-psychometric literature, taking these spaces as the primary objects to recover. To this end, we relate log co-occurrences of words in large corpora to semantic similarity assessments and show that co-occurrences are indeed consistent with an Euclidean semantic space hypothesis. Framing word embedding as metric recovery of a semantic space unifies existing word embedding algorithms, ties them to manifold learning, and demonstrates that existing algorithms are consistent metric recovery methods given co-occurrence counts from random walks. Furthermore, we propose a simple, principled, direct metric recovery algorithm that performs on par with the state-of-the-art word embedding and manifold learning methods. Finally, we complement recent focus on analogies by constructing two new inductive reasoning datasets-series completion and classification-and demonstrate that word embeddings can be used to solve them as well.
