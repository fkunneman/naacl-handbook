In recent years extracting relevant information from biomedical and clinical texts such as research articles, discharge summaries, or electronic health records have been a subject of many research efforts and shared challenges. Relation extraction is the process of detecting and classifying the semantic relation among entities in a given piece of texts. Existing models for this task in biomedical domain use either manually engineered features or kernel methods to create feature vector. These features are then fed to classifier for the prediction of the correct class. It turns out that the results of these methods are highly dependent on quality of user designed features and also suffer from curse of dimensionality. In this work we focus on extracting relations from clinical discharge summaries. Our main objective is to exploit the power of convolution neural network (CNN) to learn features automatically and thus reduce the dependency on manual feature engineering. We evaluate performance of the proposed model on i2b2-2010 clinical relation extraction challenge dataset. Our results indicate that convolution neural network can be a good model for relation exaction in clinical text without being dependent on expert's knowledge on defining quality features.
