Website privacy policies are often ignored by Internet users, because these documents tend to be long and difficult to understand. However, the significance of privacy policies greatly exceeds the attention paid to them: these documents are binding legal agreements between website operators and their users, and their opaqueness is a challenge not only to Internet users but also to policy regulators. One proposed alternative to the status quo is to automate or semi-automate the extraction of salient details from privacy policy text, using a combination of crowdsourcing, natural language processing, and machine learning. However, there has been a relative dearth of datasets appropriate for identifying data practices in privacy policies. To remedy this problem, we introduce a corpus of 115 privacy policies (267K words) with manual annotations for 23K fine-grained data practices. We describe the process of using skilled annotators and a purpose-built annotation tool to produce the data. We provide findings based on a census of the annotations and show results toward automating the annotation procedure. Finally, we describe challenges and opportunities for the research community to use this corpus to advance research in both privacy and language technologies.
