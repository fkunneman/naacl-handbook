In this paper we present a study on the production of collocations by students of European Portuguese as a foreign language. We start by gathering several corpora written by students, and identify the correct and incorrect collocations. We annotate the latter considering several different aspects, such as the error location, description and explanation. Then, taking these elements into consideration, we compare the performance of students considering their levels of proficiency, their mother tongue and, also, other languages they know. Finally, we correct all the students productions and contribute with a corpus of everyday language collocations that can be helpful in Portuguese classes.
