We present a computational analysis of three varieties of language: native, advanced non-native, and translation. Our goal is to investigate the similarities and differences between non-native language productions and translations, contrasting both with native language. Using a collection of computational methods we establish three main results: (1) the three types of texts are easily distinguishable; (2) non-native language and translations are closer to each other than each of them is to native language; and (3) some of these characteristics are source- and native-language-dependent, while others are not, reflecting, perhaps, unified principles that similarly affect translations and non-native language.
