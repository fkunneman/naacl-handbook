This paper employs both a web-as-corpus and a Twitter-as-corpus approach to present a longitudinal case study of the establishment of three recently coined, synonymous neologisms: 'rapefugee', 'rapeugee' and 'rapugee'. We describe the retrieval and processing of the web and Twitter data and discuss the dynamics of the competition between the three forms within and across both datasets based on quantitative summaries of the results. The results show that various language-external events boost the usage of the terms both on the web and on Twitter, with the latter typically ahead of the former by some days. Beside absolute frequencies, we distinguish between several special usages of the target words and their effects on the establishment process. For the web corpus, we examine target words appearing in the title of websites and metalinguistic usages; for the Twitter corpus, we examine hashtag uses and retweets. We find that the use of hashtags and retweets significantly affects the spread of the neologisms both on Twitter and on the web.
