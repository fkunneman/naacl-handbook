In this paper I present a k-means clustering approach to inferring morphological position classes (morphotactics) from Interlinear Glossed Text (IGT), data collections available for some endangered and low-resource languages. While the experiment is not restricted to low-resource languages, they are meant to be the targeted domain. Specifically my approach is meant to be for field linguists who do not necessarily know how many position classes there are in the language they work with and what the position classes are, but have the expertise to evaluate different hypotheses. It builds on an existing approach (Wax,2014), but replaces the core heuristic with a clustering algorithm. The results mainly illustrate two points. First, they are largely negative, which shows that the baseline algorithm (summarized in the paper) uses a very predictive feature to determine whether affixes belong to the same position class, namely edge overlap in the affix graph. At the same time, unlike the baseline method that relies entirely on a single feature, k-means clustering can account for different features and helps discover more morphological phenomena, e.g. circumfixation. I conclude that unsupervised learning algorithms such as k-means clustering can in principle be used for morphotactics inference, though the algorithm should probably weigh certain features more than others. Most importantly, I conclude that clustering is a promising approach for diverse morphotactics and as such it can facilitate linguistic analysis of field languages.
