Medical sciences have long since established an ethics code for experiments, to minimize the risk of harm to subjects. Natural language processing (NLP) used to involve mostly anonymous corpora, with the goal of enriching linguistic analysis, and was therefore unlikely to raise ethical concerns. As NLP becomes increasingly wide-spread and uses more data from social media, however, the situation has changed: the outcome of NLP experiments and applications can now have a direct effect on individual users' lives. Until now, the discourse on this topic in the field has not followed the technological development, while public discourse was often focused on exaggerated dangers. This position paper tries to take back the initiative and start a discussion. We identify a number of social implications of NLP and discuss their ethical significance, as well as ways to address them.
