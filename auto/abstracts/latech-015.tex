We present results on part-of-speech and morphological tagging for Old Swedish (1225---1526). In a set of experiments we look at the difference between within-corpus and across-corpus accuracy, and explore ways of mitigating the effects of variation and data sparseness by adding different types of dictionary information. Combining several methods, together with a simple approach to handle spelling variation, we achieve a major boost in tagger performance on a modest test collection.
