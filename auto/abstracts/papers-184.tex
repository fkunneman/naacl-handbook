Named Entity Disambiguation (NED) algorithms disambiguate mentions of named entities with respect to a knowledge-base, but sometimes the context might be poor or misleading. In this paper we introduce the acquisition of two kinds of background information to alleviate that problem: entity similarity and selectional preferences for syntactic positions. We show, using a generative model for NED, that the additional sources of context are complementary, and improve results in the CoNLL2003 and TAC KBP DEL 2014 datasets, yielding the third best and the best results, respectively. We provide examples and analysis which show the value of the acquired background information.
