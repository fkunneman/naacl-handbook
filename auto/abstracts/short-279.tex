Online reviews are a growing market, but it is struggling with fake reviews. They undermine both the value of reviews to the user, and their trust in the review sites. However, fake positive reviews can boost a business, and so a small industry producing fake reviews has developed. The two sides are facing an arms race that involves more and more natural language processing (NLP). So far, NLP has been used mostly for detection, and works well on human-generated reviews. But what happens if NLP techniques are used to generate fake reviews as well? We investigate the question in an adversarial setup, by assessing the detectability of different fake-review generation strategies. We use generative models to produce reviews based on meta-information, and evaluate their effectiveness against deception-detection models and human judges. We find that meta-information helps detection, but that NLP-generated reviews conditioned on such information are also much harder to detect than conventional ones.
