This paper studies how word embeddings trained on the British National Corpus interact with part of speech boundaries. Our work targets the Universal PoS tag set, which is currently actively being used for annotation of a range of languages. We experiment with training classifiers for predicting PoS tags for words based on their embeddings. The results show that the information about PoS affiliation contained in the distributional vectors allows us to discover groups of words with distributional patterns that differ from other words of the same part of speech. This data often reveals hidden inconsistencies of the annotation process or guidelines. At the same time, it supports the notion of `soft' or `graded' part of speech affiliations. Finally, we show that information about PoS is distributed among dozens of vector components, not limited to only one or two features.
