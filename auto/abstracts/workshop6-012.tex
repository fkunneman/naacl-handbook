We present a novel approach to the unsupervised learning of morphology.  In particular, we use a Multiple Cause Mixture Model (MCMM), a type of autoencoder network consisting of two node layers---hidden and surface---and a matrix of weights connecting hidden nodes to surface nodes. We show that an MCMM shares crucial graphical properties with autosegmental morphology. We argue on the basis of this graphical similarity that our approach is theoretically sound. Experiment results on Hebrew data show that this theoretical soundness bears out in practice.
