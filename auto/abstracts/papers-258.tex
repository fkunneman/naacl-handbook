Linguistic drift is a process that produces slow irreversible changes in the grammar and function of a language's constructions.  Importantly, changes in a part of a language can have trickle down effects, triggering changes elsewhere in that language. Although such causally triggered chains of changes have long been hypothesized by historical linguists, no explicit demonstration of the actual causality has been provided. In this study, we use co-occurrence statistics and machine learning to demonstrate that the functions of morphological cases experience a slow, irreversible drift along history, even in a language as conservative as is Icelandic. Crucially, we then move on to demonstrate --using the notion of Granger-causality-- that there are explicit causal connections between the changes in the functions of the different cases, which are consistent with documented processes in the history of Icelandic. Our technique provides a means for the quantitative reconstruction of connected networks of subtle linguistic changes.
