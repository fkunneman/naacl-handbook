We investigate implicit corrections in the form of contrastive discourse in child-adult interaction, which have been argued to contribute to language learning. In contrast to previous work in psycholinguistics, we adopt a data-driven methodology, using comparably large amounts of data and  leveraging computational methods. We conduct a corpus study on the use of parental corrective feedback and show that its presence in child directed speech is associated with a reduction of child subject omission errors in English.
