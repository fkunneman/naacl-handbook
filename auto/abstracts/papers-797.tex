Through a particular choice of a predicate (e.g., ``x violated y''), a writer can subtly connote a range of implied sentiment and presupposed facts about the entities x and y: (1) writer's perspective: projecting x as an ``antagonist'' and y as a ``victim'', (2) entities' perspective: y probably dislikes x, (3) effect: something bad happened to y, (4) value: y is something valuable, and (5) mental state: y is distressed by the event. We introduce connotation frames as a representation formalism to organize these rich dimensions of connotation using typed relations. First, we investigate the feasibility of obtaining connotative labels through crowdsourcing experiments. We then present models for predicting the connotation frames of verb predicates based on their distributional word representations and the interplay between different types of connotative relations. Empirical results confirm that connotation frames can be induced from various data sources that reflect how language is used in context. We conclude with analytical results that show the potential use of connotation frames for analyzing subtle biases in online news media.
