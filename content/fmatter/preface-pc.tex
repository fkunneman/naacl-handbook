
\section{Message from the Program Committee Co-Chairs}\vspace{2em}
\setheaders%
    {Message from the Program Committee Co-Chairs}%
    {Message from the Program Committee Co-Chairs}
\thispagestyle{emptyheader}

\setlength{\parskip}{1ex}

Welcome to the 54th Annual Meeting of the Association for Computational Linguistics! This year, ACL received 825 long paper submissions (a new record)
and 463 short paper submissions.\footnote{These numbers exclude papers that were not reviewed due to formatting violations or that were withdrawn prior to review.}  
Of the long papers, 231 were accepted for presentation at ACL---116 as oral presentations and 115 as poster presentations. 97 short papers were accepted---50 as oral and 47 as poster presentations. In addition, ACL also features 25 presentations of papers accepted in the \emph{Transactions of the Association for Computational Linguistics} (TACL).  With 353 papers being presented, this is the largest ACL program to date.

In keeping with the tremendous growth of our field, we introduced some changes to the conference.  First, oral presentations were shortened to fifteen (twelve) minutes for long (short) papers, plus time for questions.  While this places a greater demand on speakers to be concise, we believe it is worth the effort, allowing far more work to be presented orally.  Second, we took advantage of the many halls available at Humboldt University and expanded the number of parallel talks during some conference sessions.  

We introduced a category of outstanding papers to help recognize the highest quality work in the community this year.  The 11 outstanding papers (9 long, 2 short, 0.85\% of submissions) represent a broad spectrum of exciting contributions; they are recognized by especially prominent placement in the program.  From these, a best paper and an IBM-sponsored best student paper have been selected; those will be announced in the awards session on Wednesday afternoon.

Following other recent ACL conferences, submissions were reviewed under different categories and using different review forms for empirical/data-driven, theoretical, applications/tools, resources/evaluation, and survey papers.  We introduced special fields in the paper submission form for authors to explicitly note the release of open-source implementations to enable reproducibility, and to note freely available datasets. We also allowed authors to submit appendices of arbitrary length for details that would enable replication; reviewers were not expected to read this material.

Another innovation we explored during the review period was the scheduling of short paper review before long paper review.  While this was planned to make the entire review period more compact (fitting between the constraints of NAACL 2016 and EMNLP 2016 at either end), we found that reviewing short papers first eliminated many of the surprises for the long paper review process.

We sought to follow recently-evolved best practices in planning the poster sessions, so that the many high-quality works presented in that format will be visible and authors and attendees benefit from the interactions during the two poster sessions.

ACL 2016 will have two distinguished invited speakers: Amber Boydstun (Associate Professor of Political Science at the University of California, Davis) and Mark Steedman (Professor of Cognitive Science at the University of Edinburgh). We are grateful that they accepted our invitation and look forward to their presentations.

There are many individuals we wish to thank for their contributions to ACL 2016, some multiple times:

\begin{itemize}
\item The 38 area chairs who recruited reviewers, led the discussion about each paper, carefully assessed each submission, and authored meta-reviews to guide final decisions:
Miguel Ballesteros,
David Bamman,
Steven Bethard,
Jonathan Berant,
Gemma Boleda,
Ming-Wei Chang,
Wanxiang Che,
Chris Dyer,
Ed Grefenstette,
Hannaneh Hajishirzi,
Minlie Huang,
Mans Hulden,
Heng Ji,
Jing Jiang,
Zornitsa Kozareva,
Marco Kuhlmann,
Yang Liu,
Annie Louis,
Wei Lu,
Marie-Catherine de Marneffe,
Gerard de Melo,
David Mimno,
Meg Mitchell,
Daichi Mochihashi,
Graham Neubig,
Naoaki Okazaki,
Simone Ponzetto,
Matthew Purver,
David Reitter,
Nathan Schneider,
Hinrich Schuetze,
Thamar Solorio,
Lucia Specia,
Partha Talukdar,
Ivan Titov,
Lu Wang,
Nianwen Xue, and
Grace Yang.
\item Our full program committee of 884 hard-working individuals who reviewed the conference's 1,288 submissions (including secondary reviewers). 
\item The ACL coordinating committee members, especially Yejin Choi,  Graeme Hirst,  Chris Manning, and Shiqi Zhao, who answered many questions as they arose during the year.
\item TACL editors-in-chief Mark Johnson, Lillian Lee, and Kristina Toutanova, for coordinating with us on TACL presentations at ACL.
\item Ani Nenkova and Owen Rambow, program co-chairs of NAACL 2016, and Michael Strube, program co-chair of ACL 2015, who were generous with advice.
\item Yusuke Miyao, Yannick Versley, and Hai Zhao, our well-organized publication chairs, and the responsive team at Softconf led by Rich Gerber.
\item Valia Kordoni and the local organization team, especially webmaster Kostadin Cholakov.
\item Antal van den Bosch, our general chair, who kept us coordinated with the rest of the ACL 2016 team and offered guidance whenever we needed it.
\item Antal van den Bosch, Claire Cardie, Pascale Fung, Ray Mooney, and Joakim Nivre, who carefully reviewed papers under consideration for outstanding and best paper recognition.
\item Priscilla Rasmussen, who knows everything about how to make ACL a success.
\end{itemize}

We hope that you enjoy ACL 2016 in Berlin!

\vskip 0.5in
\noindent Katrin Erk, University of Texas \\
\noindent Noah A.~Smith, University of Washington \\ \\
Program Committee Co-Chairs

\index{Smith, Noah}
\index{Erk, Katrin}
