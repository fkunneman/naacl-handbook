\begin{bio}
{\bfseries Philipp Koehn} is a leading researcher in statistical machine translation and co-ordinated the EU-funded CASMACAT project on computer aided translation 2011-2014.
\end{bio}

\begin{tutorial}
  {Computer Aided Translation: Advances and Challenges}
  {tutorial-003}
%{Philipp Koehn}
  {\daydateyear, \tutorialmorningtime}
  {\TutLocC}

Moving beyond post-editing machine translation, a number of recent research efforts have advanced computer aided translation methods that allow for more interactivity, richer information such as confidence scores, and the completed feedback loop of instant adaptation of machine translation models to user translations. 

This tutorial will explain the main techniques for several aspects of computer aided translation: 

\begin{itemize}
  \item confidence measures
  \item interactive machine translation (interactive translation prediction)
  \item bilingual concordancers
  \item translation option display
  \item paraphrasing (alternative translation suggestions)
  \item visualization of word alignment
  \item online adaptation
  \item automatic reviewing
  \item integration of translation memory
  \item eye tracking, logging, and cognitive user models
\end{itemize}

For each of these, the state of the art and open challenges are presented. The tutorial will also look under the hood of the open source CASMACAT toolkit that is based on MATECAT, and available as a ``Home Edition'' to be installed on a desktop machine. The target audience of this tutorials are researchers interested in computer aided machine translation and practitioners who want to use or deploy advanced CAT technology. 

\end{tutorial}
