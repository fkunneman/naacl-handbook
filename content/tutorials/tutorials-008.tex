\begin{bio}
{\bfseries Miriam R. L. Petruck} received her PhD in Linguistics from the University of California, Berkeley. A key member of the team developing FrameNet almost since the project's founding, her research interests include semantics, knowledge base development, grammar and lexis, lexical semantics, Frame Semantics and Construction Grammar.

{\bfseries Ellen K. Dodge} received her PhD in Linguistics from the University of California, Berkeley. Since 2000, she has worked in ICSI's AI Group, first as part of the Neural Theory of Language project, and also FrameNet. She is the primary linguist continuing to develop Embodied Construction Grammar. Since 2012, she has worked on MetaNet, developing formal representations of frame and metaphor networks, as well as automatic methods to identify and analyze metaphors in text.
\end{bio}

\begin{tutorial}
  {MetaNet: Repository, Identification System, and Applications}
  {tutorial-008}
%{Miriam R. L. Petruck and Ellen K. Dodge}
  {\daydateyear, \tutorialafternoontime}
  {\TutLocH}

The ubiquity of metaphor in language (Lakoff and Johnson 1980) has served as impetus for cognitive linguistic approaches to the study of language, mind, and the study of mind (e.g. Thibodeau \& Boroditsky 2011). While native speakers use metaphor naturally and easily, the treatment and interpretation of metaphor in computational systems remains challenging because such systems have not succeeded in developing ways to recognize the semantic elements that define metaphor. This tutorial demonstrates MetaNet's frame-based semantic analyses, and their informing of MetaNet's automatic metaphor identification system. Participants will gain a complete understanding of the theoretical basis and the practical workings of MetaNet, and acquire relevant information about the Frame Semantics basis of that knowledge base and the way that FrameNet handles the widespread phenomenon of metaphor in language. The tutorial is geared to researchers and practitioners of language technology, not necessarily experts in metaphor analysis or knowledgeable about either FrameNet or MetaNet, but who are interested in natural language processing tasks that involve automatic metaphor processing, or could benefit from exposure to tools and resources that support frame-based deep semantic, analyses of language, including metaphor as a widespread phenomenon in human language. \\
 
%Tutorial Outline \\

%Part I: Background to FrameNet and the FrameNet Constructicon 
%\begin{enumerate}[label=(\alph*)]
%  \item Frame Semantics and FrameNet
%  \item Construction Grammar and the FrameNet Constructicon
%  \item FrameNet treatment of metaphor
%\end{enumerate}

%Part II: Overview of MetaNet
%\begin{enumerate}[label=(\alph*)]
%  \item Conceptual Metaphor Theory
%  \item The MetaNet Repository
%  \item The Metaphor Identification System 
%\end{enumerate}

%Part III: Applications
%\begin{enumerate}[label=(\alph*)]
%  \item Metaphor Identification and Analysis
%  \item Multilinguality
%  \item Information Extraction

%Part IV: Challenges and Opportunities
%\begin{enumerate}[label=(\alph*)]
%  \item Integrating FrameNet and MetaNet
%  \item Further system development of MetaNet
%\end{enumerate}

\end{tutorial}
