
\chapter{Local Guide}

This year's Association for Computational Linguistics annual meeting (ACL 2016) takes place in Berlin, Germany. Here are some things that we think might be useful or enjoyable for conference participants.

\section{Transportation}

The  city  is  served  by  two airports, Berlin Tegel and Berlin Sch\"{o}nefeld, linking it to  major cities all over the world The city is also one of the major nodes in Europe's rail and road networks. Thusit is an excellent starting point for exploring the rest of  Germany  or  its  neighboring countries. Finally, Berlin has an excellent and extensive public transportation system which operates 24 hours, 7 days a week.

\subsection{Airports}

Berlin is served by two airports, providing multiple flights a day to a huge number of European, North African and Middle Eastern destinations. There are also daily long-haul flights to a large number of destinations outside Europe. Most airlines fly from the Berlin-Tegel airport, situated 8km northwest of the city centre. Berlin-Sch\"{o}nefeld, on the other hand, is mainly used by charter and low-cost airlines such as EasyJet, Ryanair, Norwegian Air Shuttle, and Condor. This gives participants from Europe, North Africa, and Turkey the opportunity to fly directly to Berlin at very low prices. However, some legacy airlines such as Aeroflot, El Al, and TAP Portugal also use the airport. The airport is situated 18 km southeast of the city center.

\subsection{Transport from airports to city}

\paragraph{From Berlin-Tegel:} There is a direct bus line TXL which connects the airport with the main train station (Hauptbahnhof). From Hauptbahnhof, there are three urban rail lines S5, S7 and S75 to the station ``Friedrichstrasse'' which is about 5-10 minutes walk from the conference venue. The trains depart from platform 15 at the top level of the train station. The whole trip from the airport takes between 30 and 40 minutes and costs 2,70€. There are also bus lines which connect the airport to other parts of Berlin for the same price. Depending on the time of day, the buses run every 6 to 15 minutes. Taxis are available outside the airport and a trip to the city center would typically cost around 25 €. The trip takes between 20 and 30 minutes, depending on the traffic conditions.

\paragraph{From Berlin-Sch\"{o}nefeld:} Between 4.30h. and 23h an airport express train travels in 30 minute intervals between Sch\"{o}nefeld and the city center. It provides a direct connection to the conference venue (station ``Friedrichstrasse'') which can be reached within 25 minutes. The ticket to the city costs 3,20€. There are also two urban rail trains S9 and S45 which connect the airport to the southern and eastern parts of Berlin. Those have longer operating hours and the city center can be easily reached with a single change to the Berlin metro system. In this case the trip takes a bit longer but the cost of the ticket remains the same. Taxis are also available at a price of about 40€.

\subsection{Train Transport}

Berlin is a major hub of the Deutsche Bahn, the German national railway company. There are direct high speed train connections to all major German cities as well as to all neighboring countries. For example, the trip to Hamburg takes about 1.5 hours to Frankfurt about 3.5 hours and to Cologne about 4.5 hours. The new central station was opened in 2006 and it is a 5 minute walk from it to the Reichstag building and the Brandenburg Tor. This makes the station an excellent starting point for exploring Berlin. It is also connected to both airports by direct bus and train connections. Most trains, including high speed ones, also stop at other train stations within Berlin, so one can get on and off at another, more convenient location. Finally, the major German train stations feature extensive markings and information in English. Furthermore, written and voice information in English is given in all high speed trains. Therefore, conference participants can get around easily without knowing any German.

 \subsection{City transport}

 Berlin's local public transport network consists of city train/urban rail (S-Bahn), metro (U-Bahn), buses and trams. There are three zones: AB, BC and ABC.

The conference participants will need an AB or ABC ticket -- AB ticket covers the whole Berlin area until the city boundary. However, Sch\"{o}nefeld Airport and Potsdam are outside of the Berlin city boundary, i.e. in the zone C. Therefore, participants intending to go to zone C are advised to buy the ABC ticket.

There are various types of tickets offered:
\begin{itemize}
  \item short distance tickets -- up to 3 urban rail or metro stations (change allowed) and up to 6 bus or tram stations (no change)
  \item single tickets
  \item daily tickets
  \item 7-day tickets
  \item and some more.
\end{itemize}

We recommend a 7-day ticket:
\begin{itemize}
  \item for zones AB: 30 €
  \item for zones ABC: 37.20 € (includes Sch\"{o}nefeld Airpor and  Potsdam)
\end{itemize}

If you fly from Sch\"{o}nefeld Airport or you would like to visit Potsdam, it will be more convenient for you to by the ABC ticket. With the 7-day ticket, on Saturdays, Sundays as well as on working days between 20h and 3h, you can bring with you one adult person and up to three 6 to 14 years old kids.

\paragraph{Buying tickets:}
Tickets can be purchased at multilingual ticket machines on the platforms of S-and U-Bahn stations. In buses, fares are paid to the bus driver, in trams at machines inside the trains. In larger stations the S-Bahn and BVG provide ticket counters.

\paragraph{Validation of tickets:}
Before the journey starts tickets must be validated by stamping them at the yellow or red boxes found on the platforms of all S-Bahn and U-Bahn stations. If you travel by bus or tram, there are also validating machines inside the vehicles. In case of inspection, a ticket that is not stamped is invalid. \textbf{Important}: if you have a daily or a 7-day ticket, you need to validate it only once, namely before your first journey.

\paragraph{Fare evasion:}
Anyone caught without a valid ticket must pay a higher fare of 60€. Even people who forgot to stamp their ticket must pay the fine. Note: Ticket inspectors are dressed in plain clothes and will not make any exceptions for tourists. Those who get caught have to show an ID, otherwise the police will be called.

Stops closest to the HU Berlin building are:
\begin{itemize}
  \item S-Bahn: ``S+U Friedrichstrasse'' (S-Bahn lines S5, S7, S75, S1, S2, S25)
  \item U-Bahn: ``U Friedrichstrasse, U Franz\"{o}ssische Strasse'' (U6)
  \item Bus: ``Staatsoper'' (lines 100, 200)
  \item Tram:''Universit\"{a}tstrasse'' (lines 12, M1)
\end{itemize}

\subsection{Bicycles}

Berlin also has the reputation of having excellent infrastructure for bicycles. The city features an extensive network of bicycle alleys and dedicated bicycle lanes as well as bicycle traffic lights. Bikers are also allowed to use the dedicated bus lanes at major streets. There are bicycle parking facilities throughout the city, including one at the conference venue.

Renting a bicycle in the summer is a must-do in Berlin -- there are several bicycle renting places close to the conference venue. For planning a route from place A to place B (via C), visit a web site ``Berlin By Bike'' \url{http://www.bbbike.de}

It is also possible to take a bicycle in marked cars of S-Bahn, U-Bahn and tram. However, an additional bicycle ticket has to be purchased.

\section{Food}

There is a huge number of locations for having lunch or dinner within walking distance of the conference venue, ranging from simple bakeries and sandwich bars to gourmet restaurants. Since the area contains major tourist attractions, virtually all food locations provide menus in multiple languages.

International fast food chains such as McDonalds, Burger King, and Subway are located within walking distance. Additionally, the streets round the venue are full with local sandwich and salad bars, traditional German bakeries, doner and kebab shops, and Asian fast food places. It is also worth trying the traditional Berlin `Curry Wurst' (curry sausage) sold by street sellers. 

A number of restaurants from all price categories are also found in the surrounding area. Those offer traditional German  as well as international cuisine. Many restaurants offer the so called `business lunch' which tip cally includes a main dish and a drink for a fixed price. The prices of such a lunch usually start from 6€.

Last but not least, Berlin is famous for its large `green' community. A lot of restaurants, bakeries, and sandwich  bars offer food prepared only with biological ingredients. It is also easy to find vegetarian and vegan dishes and those dishes are clearly indicated in the menus of all restaurants.

Below, we list some of the venues located nearby the conference site together with the approximate prices of business lunch offers or those  of the main dishes.

\begin{tabular}{|c|c|c|c|}
%  \toprule
  Name & Type of food & Distance & Prices \\ \hline
 % \midrule
  Cum laude & German and international & on site & from 6€ \\
  12 Apostel & pizza and pasta & behind the venue & business lunch, 7-10€ \\
  Deponie No. 3 & German and Berlin & behind the venue & 5-15€ \\
  Via Nova & Italian & behind the venue & business lunch, 7-11€ \\
  Leon & caf\'{e} and restaurant & 3 min walking & 5-15€ \\
  Restaurant Nolle & international and Berlin & 3 min walking & 5-15€ \\
  Die Maultasche & Swabian delicacies & 3 min walking & 5-9€ \\
  M\"{o}venpick Restaurant & small international menu & 2 min walking & 6-13€ \\
  Daimlers & international & 2 min walking & 7-20€ \\
  Vapiano & Italian, self-service & 4 min walking& moderate \\
  Suppenb\"{o}rse & soups & 3 min walking & 3-6 eusro \\
  Casa Italia & pizza, salads, pasta & 4 min walking & 6-16€ \\
  Block House & steak house & 4 min walking & 9-12€ \\
  Subway & fast food, sandwiches & 4 min walking & low, moderate \\
  Meyerbeers Coffee & caf\'{e} & 3 min walking & moderate \\
  Odeon & caf\'{e}, snacks, mainly cakes & behind the venue & moderate \\
  Starbucks & caf\'{e}, snacks and cakes & 5 min walking & moderate \\
  Kofler Caf\'{e} & caf\'{e}, snacks and meals & 2 min walking & 5-20€ \\
  McDonalds & fast food & 4 min walking & low, moderate \\
  Burger King & fast food & 4 min walking & low, moderate \\
  Asia gourmet & Asian take-away & 4 min walking & low \\
  Kl\"{a}ssig's Fish \& Chips & fish fast food & 4 min walking & low \\
  Damisch & fast food & 4 min walking & low \\
  McCornells Obsttresen & fruit, snacks, fast food & 4 min walking & low \\
  Fischers Fritz*** & German and international & 10 min walking & 25-48€ \\
  Dressler Restaurant & national and international & 6 min walking &  16-30€ \\
  Bocca di Bacco & Italian & 5 min walking & 17-35€ \\
%  \bottomrule
\end{tabular}

\section{Weather}

Summer in Berlin can have many faces, from rainy and windy ~15° to sunny and hot ~4°C. Check the weather forecast 2-3 days before the trip, for example here: \url{http://www.wetteronline.de/wetter/berlin} (Berlin weather forecast)

\section{Safety}

Berlin is generally a safe city. Nevertheless, be aware of pick-pocketing, especially at large S-Bahn and U-Bahn stations. In addition, do not sign any papers, petitions or similar offered by random people at the street or in the public transport.
